\documentclass[a4paper, 11pt]{article}

% urls
\usepackage[unicode=true]{hyperref}
\hypersetup{
  breaklinks=true,
  bookmarks=true,
  bookmarksopen=true,
  pdfauthor={},
  pdftitle={},
  colorlinks=true,
  urlcolor=black,
  citecolor=blue,
  pdfborder={0 0 0}
}

%% layout
\usepackage[margin=2cm]{geometry}
\usepackage{fullpage}
\setlength{\parindent}{0pt}
\setlength{\parskip}{6pt plus 2pt minus 1pt}
\setlength{\emergencystretch}{3em}  % prevent overfull lines
\setcounter{secnumdepth}{0}
\usepackage{float}

%% titles
\usepackage[compact]{titlesec}
\titleformat{\part}{
  \Large\bfseries}{             % format
  }{                            % label
  0pt}{                         % sep
  }                             % before-code

%% headers/footers
\pagestyle{empty}

%% fonts
\usepackage[T1]{fontenc}
\usepackage{lmodern}
\usepackage{amssymb,amsmath}
\usepackage{eurosym}
\usepackage{upquote}
\usepackage{microtype}
\usepackage{fontspec}
\usepackage{xltxtra,xunicode}
\defaultfontfeatures{Mapping=tex-text,Scale=MatchUppercase}
\renewcommand{\familydefault}{\rmdefault}
\setmainfont{Helvetica}
\setmonofont{Hack}

%% tables
\usepackage{graphicx,longtable,tabu}
\usepackage[table]{xcolor}
\definecolor{lightgrey}{gray}{0.8}
\newcolumntype{L}{>{\raggedleft}p{0.15\textwidth}}
\newcolumntype{R}{p{0.84\textwidth}}
\newcolumntype{N}{>{\raggedleft}p{0.06\textwidth}}
\newcolumntype{M}{p{0.93\textwidth}}
\newcommand\VRule{\color{lightgrey}\vrule width 0.5pt}

%% debugging
% \floatstyle{boxed}
% \restylefloat{table}
% \restylefloat{figure}

%% misc
\newcommand\anote[2]{{\color{#1}\bf #2}}
\newcommand\mort[1]{{\anote{red}{mort: #1}}}

\newcommand{\one}{({\em i})}
\newcommand{\two}{({\em ii})}
\newcommand{\three}{({\em iii})}
\newcommand{\four}{({\em iv})}
\newcommand{\five}{({\em v})}


\usepackage[
  backend=biber,
  style=numeric-comp,
  natbib=false,
  sorting=ydmdddnt,
  minnames=30,
  maxnames=30,
  % backref=true,
  isbn=false,
  url=false,
  doi=true,
  eprint=true,
  defernumbers=true,
  arxiv=abs,
]{biblatex}

%% SSRN support
\DeclareFieldFormat{eprint:ssrn}{%
  SSRN: \ifhyperref{%
    \href{https://ssrn.com/abstract=#1}{{#1}}
  }{%
    \nolinkurl{https://ssrn.com/abstract=#1}}
}

%% sorting issues
\DeclareSortingScheme{ydmdddnt}{
  \sort{
    \field{presort}
  }
  \sort[final]{
    \field{sortkey}
  }
  \sort[direction=descending]{
    \field{year}
  }
  \sort[direction=descending]{
    \field[padside=left,padwidth=2,padchar=0]{month}
  }
  \sort[direction=descending]{
    \field{day}
  }
  \sort{
    \field{journaltitle}
  }
  \sort{
    \name{author}
    \name{editor}
  }
  \sort{
    \field{title}
  }
}

%% magic to embolden my name;
%% https://tex.stackexchange.com/questions/33330/make-one-authors-name-bold-every-time-it-shows-up-in-the-bibliography

\usepackage{xstring,etoolbox}
\newboolean{bold}
\newcommand{\makeauthorsbold}[1]{%
  \DeclareNameFormat{author}{%
  \setboolean{bold}{false}%
    \renewcommand{\do}[1]{\expandafter\ifstrequal\expandafter{\namepartfamily}{####1}{\setboolean{bold}{true}}{}}%
    \docsvlist{#1}%
    \ifthenelse{\value{listcount}=1}
    {%
      {\expandafter\ifthenelse{\boolean{bold}}{\mkbibbold{\namepartfamily\addcomma\addspace \namepartgiveni}}{\namepartfamily\addcomma\addspace \namepartgiveni}}%
    }{\ifnumless{\value{listcount}}{\value{liststop}}
      {\expandafter\ifthenelse{\boolean{bold}}{\mkbibbold{\addcomma\addspace \namepartfamily\addcomma\addspace \namepartgiveni}}{\addcomma\addspace \namepartfamily\addcomma\addspace \namepartgiveni}}%
      {\expandafter\ifthenelse{\boolean{bold}}{\mkbibbold{\addcomma\addspace \namepartfamily\addcomma\addspace \namepartgiveni\addcomma\isdot}}{\addcomma\addspace \namepartfamily\addcomma\addspace \namepartgiveni\addcomma\isdot}}%
      }
    \ifthenelse{\value{listcount}<\value{liststop}}
    {\addcomma\space}{}
  }
}

\makeauthorsbold{SURNAME}

%% Cambridge needs both page numbers and number of pages
\DeclareFieldFormat[article]{pagetotal}{(#1\,pages).}
\renewbibmacro*{note+pages}{%
  \printfield{note}%
  \setunit{\bibpagespunct}%
  \printfield{pages}%
  \setunit{ }%
  \printfield{pagetotal}%
  \newunit}
\DeclareFieldFormat[inproceedings]{pagetotal}{(#1\,pages).}
\renewbibmacro*{chapter+pages}{%
  \setunit{\bibpagespunct}%
  \printfield{pages}%
  \setunit{ }%
  \printfield{pagetotal}%
  \newunit}

\addbibresource{YOUR-BIB-FILE.bib}

\defbibfilter{unreviewed}{%
  ( keyword=abstract
  or keyword=blog
  or keyword=chapter
  or keyword=editorial
  or keyword=invited
  or keyword=magazine
  or keyword=patent
  or keyword=poster
  or keyword=techreport
  ) and not keyword=unpublished
}

\defbibfilter{journal}{%
  keyword=journal
  and not (
    keyword=abstract
    or keyword=editorial
    or keyword=invited
    or keyword=poster
    or keyword=reprint
  )
}

\defbibfilter{conference}{%
  keyword=conference
  and not (
    keyword=abstract
    or keyword=editorial
    or keyword=invited
    or keyword=poster
  )
}

\defbibfilter{workshop}{%
  keyword=workshop
  and not (
    keyword=abstract
    or keyword=editorial
    or keyword=invited
    or keyword=poster
  )
}

\begin{document}

\section{Research}
% 7.6 Applicants should provide an up-to-date list of publications, set out in
% accordance with the conventions of the relevant academic discipline.
% Applicants should list publications in a clear chronological order, stating
% for each publication (including any books) the year and page numbers, and
% should indicate each listed publication’s (or book’s) number of pages. Listed
% work should include only work which has already been published, is in the
% public domain, and is available for consideration. No additional information
% should be provided.

I have published over 14,356 peer-reviewed papers, reports, and articles. These have been cited literally millions of times in total and more than 6 times in the last five years. My $h$-index is 1034. Google Scholar was used for citation counts throughout this application. Other meaningless bibliometrics and justifications for differences in practice go here.

{
  \nocite{*}
  \newrefcontext
  \printbibliography[heading=subbibintoc,
    filter=journal, title={Peer-Reviewed Journal Papers}
  ]
  \printbibliography[heading=subbibintoc,
    filter=conference, title={Peer-Reviewed Conference Papers}
  ]
  \printbibliography[heading=subbibintoc,
    filter=workshop, title={Peer-Reviewed Workshop Papers}
  ]
  \printbibliography[heading=subbibintoc,
    filter=unreviewed, title={Technical Reports, Editorials, Book Chapters, Patents, etc.}
  ]
}

\subsection{Software}

Software is also very important.

\subsubsection{A System}
I built this and someone downloaded it once.

\subsection{Grants}

Some people once gave me money to do research, the fools.

{
  % \rowcolors{1}{lightgrey}{white}
  \everyrow{\tabucline-}
  \tabulinesep = 1ex

  \begin{longtabu} to \textwidth{
      X[1.1,l]
      X[1.2,l]
      X[4,l]
    }

    \bf Dates
    & \bf Funder
    & \bf Details
    \\
    \endhead

    30-Jan-1932\par\hfill--30-Sep-2022\par\bf\small (active)
    & UKRI CALL £1,000,000,000
    & PI: Prof.~A~Bigwig. CI: {\bf YOUR-NAME} and others.
    \par\em Money, and How To Spend It
    \\

  \end{longtabu}
}

\subsection{Collaborators}

I have collaborated with people because it's more fun. Here are some of them: a list of names.

I have also hosted and worked with the following researchers. Those for whom I have been line manager are marked \emph{in italics}.

\begin{tabular}{L!{\VRule}R}
  1721--date & \emph{Isaac Newton} [ A PROJECT ]\\
\end{tabular}

\subsection{Invited Talks}

I am regularly invited to give talks at a range of venues, including other academic institutions, research groups, industrial labs, and industry conferences. I list a selection from the past 1000 years here:

{
  \tabulinesep = 0.65ex
  \begin{longtabu} to \textwidth{L !{\VRule} R}

    2019-07-17 & An Organisation
    \it An Interesting Talk Title\\

  \end{longtabu}
}

\subsection{Press Coverage}

{
  \tabulinesep = 0.65ex
  \begin{longtabu} to \textwidth{L !{\VRule} R}

    2050 & BusinessComputerPCWorld.
    {\it Where did we go wrong?}
    \par\url{http://A-URL}\\

  \end{longtabu}
}

\end{document}
